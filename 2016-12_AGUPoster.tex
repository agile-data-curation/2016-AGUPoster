%==============================================================================
%== template for LATEX poster =================================================
%==============================================================================
%
%--A0 beamer slide-------------------------------------------------------------
\documentclass[final]{beamer}
\usepackage[orientation=landscape,size=a1,
            scale=.85         % font scale factor
           ]{beamerposter}
%\setlength{\paperwidth}{33.1in}
%\setlength{\paperheight}{23.4in}
\setlength{\paperwidth}{60in}
\setlength{\paperheight}{36in}
           
\geometry{
  hmargin=2.5cm, % little modification of margins
}

%
\usepackage[utf8]{inputenc}

\linespread{1.15}
%
%==The poster style============================================================
\usetheme{sharelatex}

%==Title, date and authors of the poster=======================================
\title
[American Geophysical Union Annual Meeting. San Francisco, CA. December
12-16, 2016.] % Conference
{ % Poster title
Distilling Design Patterns From Agile Curation Case Studies (IN41A-1656)
}

\author{ % Authors
Karl Benedict\inst{1} \and W. Christopher Lenhardt\inst{2} \and Joshua Young\inst{3}}

\institute{
\inst{1} University of New Mexico
\inst{2} Renaissance Computing Institute
\inst{3} University Corporation for Atmospheric Research
}


\date{December 15, 2016}



\begin{document}
\begin{frame}[t]
%==============================================================================
\begin{multicols}{3}

%--abstract-------------------------------------------------------------

\subsection{Abstract}

In previous work the authors have argued that there is a need to take a
new look at the data management lifecycle. Our core argument is that the
data management lifecycle needs to be in essence deconstructed and
rebuilt. As part of this process we also argue that much can be gained
from applying ideas, concepts, and principles from agile software
development methods. To be sure we are not arguing for a rote
application of these agile software approaches, however, given various
trends related to data and technology, it is imperative to update our
thinking about how to approach the data management lifecycle, recognize
differing project scales, corresponding variations in structure, and
alternative models for solving the problems of scientific data curation.
In this paper we will describe what we term agile curation design
patterns, borrowing the concept of design patterns from the software
world and we will present some initial thoughts on agile curation design
patterns as informed by a sample of data curation case studies solicited
from participants in agile data curation meeting sessions conducted in
2015-16.

%--End of abstract------------------------------------------------------




%==============================================================================
%==The poster content==========================================================
%==============================================================================

\section{Introduction}\label{introduction}

The challenges that must be addressed by current research data
management and curation processes and strategies consist of a
combination of established practices that are not compatible with
increasing complexity in the data management landscape at the project
level; increasing expectations by sponsors, publishers, and institutions
relating to data management and curation; and rapid growth in the
volume, variety and velocity (three dimensions commonly used to define
``big data'') of data generated by and used in research. In combination
these challenges translate into an increasing need to develop effective
data management and curation strategies that align with a set of
\emph{shared values and principles} that inform management and curation
objectives, and implement processes that are \emph{well documented and
portable} across specific data management projects. It is this latter
requirement that is addressed in this poster - the development of a
framework for capturing elements of successful data curation activities
and generalizing those elements into linkages with existing design
patterns, or defining new design patterns when they don't exist.

\subsection{Work to Date}\label{work-to-date}

Thus far the focus of the project's work has been on developing a
framework within which the team can discuss the concept of \emph{agile
data curation} with the community, and iteratively evolving that
framework through a series of meeting sessions, workshops and
presentations that have been given at multiple venues including AGU
(2014, 2015), ESIP Federation Meeting (2016), Research Data Alliance
(2014, 2015, 2016), and SciDataCon (2016). In these various activities
the team has worked on communicating the conceptual framework for our
vision of agile data curation, presented a variety of initial values and
principles derived from those defined in the \emph{Manifesto for Agile
Software Development} (\emph{1}), and solicited the presentation of data
management projects that exemplify (either intentionally or
unintentionally) these principles.

\section{Process: Values -\textgreater{} Practice -\textgreater{} Design
Patterns}\label{process-values---practice---design-patterns}

While this outreach and community engagement work described above is
continuing, the work presented here is the starting point for our third
goal of adapting the concept of design patterns that had been developed
for object oriented software development (\emph{2}), and extended into
related domains (\emph{3}--\emph{7}), for use in documenting
\emph{named} data curation \emph{problems}, \emph{solutions}, and
\emph{consequences} that provide \emph{descriptions of generalized data
components that are customized to solve a general design problem in a
particular context} (adapted from (\emph{2})).

\begin{figure}[htbp]
\centering
\includegraphics{placeholder.png}
\caption{Information flow into developed design patterns.}
\end{figure}

\section{Conceptual Model for Agile Data Curation Design
Patterns}\label{conceptual-model-for-agile-data-curation-design-patterns}

Vivamus efficitur eros et luctus porttitor. Aenean non urna semper,
sollicitudin odio sed, ullamcorper augue. Ut malesuada lorem tortor, a
posuere urna tempus at. Suspendisse a lorem odio. Integer non metus eu
lacus maximus malesuada vel sit amet purus. Maecenas lacinia nisl in
justo pretium cursus sed sed mauris. Pellentesque habitant morbi
tristique senectus et netus et malesuada fames ac turpis egestas.
(\emph{2})

\begin{figure}[htbp]
\centering
\includegraphics{placeholder.png}
\caption{Proposed \emph{Agile Data Curation} design pattern elements.}
\end{figure}

Ut blandit nisl est, sit amet tempus odio pellentesque vitae. Sed auctor
ornare diam, sit amet vehicula massa tristique in. Morbi mollis elit
risus, id pretium dui gravida varius. Curabitur quis tristique odio.
Vivamus nec erat non turpis faucibus luctus. Nunc a ante vitae massa
commodo semper sed vel dui. Nulla accumsan odio et diam vestibulum
viverra. Curabitur sed lorem eget velit feugiat cursus at cursus mauris.
Nam dapibus nisl non quam maximus gravida. Integer semper cursus urna id
ultricies. Cras imperdiet enim quis augue maximus, eu condimentum diam
facilisis. Etiam et massa sodales elit sollicitudin auctor quis vitae
purus. Maecenas vitae lacus tortor. Vestibulum rhoncus congue
ullamcorper. Donec in aliquam nibh, ac suscipit elit. Lorem ipsum dolor
sit amet, consectetur adipiscing elit.

\section{Illustration of the Design Pattern Conceptual Model to a
Developed Data Management, Discovery and Access Platform -
GSToRE}\label{illustration-of-the-design-pattern-conceptual-model-to-a-developed-data-management-discovery-and-access-platform---gstore}

\begin{figure}[htbp]
\centering
\includegraphics{placeholder.png}
\caption{The Geographic Storage, Transformation and Retrieval Engine
(GSToRE) Platform .}
\end{figure}

\begin{figure}[htbp]
\centering
\includegraphics{placeholder.png}
\caption{Mapping of the GSToRE Platform's Capabilities into a Set of
Design Patterns.}
\end{figure}

Duis mauris urna, vulputate ac ex vitae, rhoncus lobortis nisi.
Vestibulum ante ipsum primis in faucibus orci luctus et ultrices posuere
cubilia Curae; In dignissim purus lacus, ut ultrices eros varius et.
Mauris dictum cursus diam, vitae pharetra felis tincidunt pharetra.
Suspendisse sem enim, lacinia eget consectetur ut, tempus vel erat.
Proin vitae enim sit amet urna elementum rutrum. Nullam consequat eros
sit amet est vestibulum rhoncus. Donec ullamcorper tempor finibus. Proin
vestibulum nulla ut metus pretium ultricies. Vestibulum diam enim,
laoreet eu luctus nec, dignissim sit amet nibh. Aenean et imperdiet
turpis, suscipit posuere orci. Phasellus sed velit maximus, facilisis
magna vitae, ornare elit. In hac habitasse platea dictumst. Ut non lacus
eu tortor varius aliquet. Sed eu urna blandit, porta nibh in, mattis
nibh.

\section{Conclusions}\label{conclusions}

Sed auctor nisl elementum leo eleifend, vel tincidunt eros vehicula.
Duis consectetur augue erat, a bibendum nisi efficitur ac. Maecenas in
justo vitae velit efficitur gravida ac in nisl. Pellentesque sit amet
nunc magna. Aliquam mollis vulputate scelerisque. Aenean et scelerisque
tellus. Donec et metus ante. Mauris dapibus lectus eget leo pretium, sit
amet pharetra urna blandit. Suspendisse scelerisque ante mi, quis
imperdiet quam suscipit quis. Mauris felis augue, rutrum a dignissim
non, semper eu turpis. Donec tempus neque non dignissim dignissim. Duis
semper ante risus. Vivamus id consectetur lectus. In enim odio, iaculis
et libero ac, aliquam dapibus mauris. Cras eu velit ornare, tincidunt
dolor quis, pellentesque nulla.

\section{Bibliography}\label{bibliography}

~

1. K. Beck \emph{et al.}, Manifesto for Agile Software Development
(2001).

2. E. Gamma, \emph{Design patterns: Elements of reusable object-oriented
software} (Addison-Wesley, Reading, Mass., 1995), \emph{Addison-wesley
professional computing series; addison-wesley professional computing
series.}

3. R. Daigneau, \emph{Service Design Patterns: Fundamental Design
Solutions for SOAP/WSDL and RESTful Web Services} (Addison-Wesley
Professional, 2011).

4. C. G. Lasater, \emph{Design Patterns} (Jones \& Bartlett Learning,
2010).

5. L. Ackerman, C. Gonzalez, \emph{Patterns-Based Engineering:
Successfully Delivering Solutions via Patterns} (Addison-Wesley
Professional, 2010).

6. A. Schwinn, J. Schelp, Design patterns for data integration.
\emph{Journal of Enterprise Information Management}. \textbf{18},
471--482 (2005).

7. G. Hohpe, B. Woolf, \emph{Enterprise Integration Patterns: Designing,
Building, and Deploying Messaging Solutions} (Addison-Wesley
Professional, 2003).

%==============================================================================
%==End of content==============================================================
%==============================================================================

%--Acknowledgements-------------------------------------------------------------

\subsection{Acknowledgements}

This work has been partially supported through funding from the National
Science Foundation (\#IIA-1301346)

%--End of Acknowledgements------------------------------------------------------


%--References------------------------------------------------------------------



%--End of references-----------------------------------------------------------

\end{multicols}

%==============================================================================
\end{frame}
\end{document}
