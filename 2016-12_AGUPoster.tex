%==============================================================================
%== template for LATEX poster =================================================
%==============================================================================
%
%--A0 beamer slide-------------------------------------------------------------
\documentclass[final]{beamer}
\usepackage[orientation=landscape,size=a1,
            scale=.85         % font scale factor
           ]{beamerposter}
%\setlength{\paperwidth}{33.1in}
%\setlength{\paperheight}{23.4in}
\setlength{\paperwidth}{60in}
\setlength{\paperheight}{36in}
           
\geometry{
  hmargin=2.5cm, % little modification of margins
}

%
\usepackage[utf8]{inputenc}

\linespread{1.15}
%
%==The poster style============================================================
\usetheme{sharelatex}

%==Title, date and authors of the poster=======================================
\title
[American Geophysical Union Annual Meeting. San Francisco, CA. December
12-16, 2016.] % Conference
{ % Poster title
Distilling Design Patterns From Agile Curation Case Studies (IN41A-1656)
}

\author{ % Authors
Karl Benedict\inst{1} \and W. Christopher Lenhardt\inst{2} \and Joshua Young\inst{3}}

\institute{
\inst{1} University of New Mexico
\inst{2} Renaissance Computing Institute
\inst{3} University Corporation for Atmospheric Research
}


\date{December 15, 2016}



\begin{document}
\begin{frame}[t]
%==============================================================================
\begin{multicols}{3}

%--abstract-------------------------------------------------------------

\subsection{Abstract}

In previous work the authors have argued that there is a need to take a
new look at the data management lifecycle. Our core argument is that the
data management lifecycle needs to be in essence deconstructed and
rebuilt. As part of this process we also argue that much can be gained
from applying ideas, concepts, and principles from agile software
development methods. To be sure we are not arguing for a rote
application of these agile software approaches, however, given various
trends related to data and technology, it is imperative to update our
thinking about how to approach the data management lifecycle, recognize
differing project scales, corresponding variations in structure, and
alternative models for solving the problems of scientific data curation.
In this paper we will describe what we term agile curation design
patterns, borrowing the concept of design patterns from the software
world and we will present some initial thoughts on agile curation design
patterns as informed by a sample of data curation case studies solicited
from participants in agile data curation meeting sessions conducted in
2015-16.

%--End of abstract------------------------------------------------------




%==============================================================================
%==The poster content==========================================================
%==============================================================================

\section{Introduction}\label{introduction}

Lorem ipsum dolor sit amet, consectetur adipiscing elit. Integer porta
ullamcorper efficitur. Mauris at nunc eu lorem viverra interdum. Aliquam
nec posuere elit. Nullam fermentum, mauris at rutrum vestibulum, est
tellus condimentum erat, at varius metus arcu non libero. Sed placerat,
risus in sagittis ornare, turpis tortor tempor sapien, vel tristique
tellus nunc eget elit. Aenean sagittis mauris ligula, in tincidunt urna
eleifend sed. Donec porttitor vestibulum magna, sed pulvinar sapien
pharetra quis. Suspendisse pharetra hendrerit tincidunt. Ut condimentum
placerat leo, eget pellentesque velit ultrices ac. Lorem ipsum dolor sit
amet, consectetur adipiscing elit.

\section{Work to Date}\label{work-to-date}

Ut maximus vehicula est et facilisis. Nunc porttitor ex eu arcu
fringilla lacinia. Curabitur ante metus, ornare nec scelerisque quis,
varius in velit. Sed vulputate iaculis massa non commodo. Maecenas a
porta tellus. Nullam eu metus condimentum, consequat lorem quis,
tincidunt lorem. Cras dignissim tincidunt tincidunt. Nunc interdum
faucibus rutrum. Aliquam a ex nisl. Fusce vehicula ac est in
condimentum. Aliquam erat volutpat. Nam sed nisi pretium, pretium arcu
tempor, cursus metus. Aliquam id tempus quam, sit amet dapibus nulla.
Nulla facilisi.

\section{Process: Values -\textgreater{} Practice -\textgreater{} Design
Patterns}\label{process-values---practice---design-patterns}

Donec varius nibh est, in consectetur nunc pharetra ac. Morbi id leo
tempus, consectetur diam in, iaculis dolor. Suspendisse volutpat viverra
tortor eget pulvinar. Aliquam fringilla ultricies lectus, id semper
diam. Suspendisse ultrices nibh nec est porta, feugiat aliquam dui
fermentum. Aenean pulvinar tellus sed lacinia hendrerit. Nulla
condimentum eget quam quis sollicitudin. Donec luctus sollicitudin quam.
Fusce nec iaculis arcu.

\begin{figure}[htbp]
\centering
\includegraphics{placeholder.png}
\caption{Information flow into developed design patterns.}
\end{figure}

\section{Conceptual Model for Agile Data Curation Design
Patterns}\label{conceptual-model-for-agile-data-curation-design-patterns}

Vivamus efficitur eros et luctus porttitor. Aenean non urna semper,
sollicitudin odio sed, ullamcorper augue. Ut malesuada lorem tortor, a
posuere urna tempus at. Suspendisse a lorem odio. Integer non metus eu
lacus maximus malesuada vel sit amet purus. Maecenas lacinia nisl in
justo pretium cursus sed sed mauris. Pellentesque habitant morbi
tristique senectus et netus et malesuada fames ac turpis egestas.
(Gamma, 1995)

\begin{figure}[htbp]
\centering
\includegraphics{placeholder.png}
\caption{\emph{Agile Data Curation} design pattern elements.}
\end{figure}

Ut blandit nisl est, sit amet tempus odio pellentesque vitae. Sed auctor
ornare diam, sit amet vehicula massa tristique in. Morbi mollis elit
risus, id pretium dui gravida varius. Curabitur quis tristique odio.
Vivamus nec erat non turpis faucibus luctus. Nunc a ante vitae massa
commodo semper sed vel dui. Nulla accumsan odio et diam vestibulum
viverra. Curabitur sed lorem eget velit feugiat cursus at cursus mauris.
Nam dapibus nisl non quam maximus gravida. Integer semper cursus urna id
ultricies. Cras imperdiet enim quis augue maximus, eu condimentum diam
facilisis. Etiam et massa sodales elit sollicitudin auctor quis vitae
purus. Maecenas vitae lacus tortor. Vestibulum rhoncus congue
ullamcorper. Donec in aliquam nibh, ac suscipit elit. Lorem ipsum dolor
sit amet, consectetur adipiscing elit.

\section{Illustration of the Design Pattern Conceptual Model to a
Developed Data Management, Discovery and Access Platform -
GSToRE}\label{illustration-of-the-design-pattern-conceptual-model-to-a-developed-data-management-discovery-and-access-platform---gstore}

\begin{figure}[htbp]
\centering
\includegraphics{placeholder.png}
\caption{The Geographic Storage, Transformation and Retrieval Engine
(GSToRE) Platform .}
\end{figure}

\begin{figure}[htbp]
\centering
\includegraphics{placeholder.png}
\caption{Mapping of the GSToRE Platform's Capabilities into a Set of
Design Patterns.}
\end{figure}

Duis mauris urna, vulputate ac ex vitae, rhoncus lobortis nisi.
Vestibulum ante ipsum primis in faucibus orci luctus et ultrices posuere
cubilia Curae; In dignissim purus lacus, ut ultrices eros varius et.
Mauris dictum cursus diam, vitae pharetra felis tincidunt pharetra.
Suspendisse sem enim, lacinia eget consectetur ut, tempus vel erat.
Proin vitae enim sit amet urna elementum rutrum. Nullam consequat eros
sit amet est vestibulum rhoncus. Donec ullamcorper tempor finibus. Proin
vestibulum nulla ut metus pretium ultricies. Vestibulum diam enim,
laoreet eu luctus nec, dignissim sit amet nibh. Aenean et imperdiet
turpis, suscipit posuere orci. Phasellus sed velit maximus, facilisis
magna vitae, ornare elit. In hac habitasse platea dictumst. Ut non lacus
eu tortor varius aliquet. Sed eu urna blandit, porta nibh in, mattis
nibh.

\section{Conclusions}\label{conclusions}

Sed auctor nisl elementum leo eleifend, vel tincidunt eros vehicula.
Duis consectetur augue erat, a bibendum nisi efficitur ac. Maecenas in
justo vitae velit efficitur gravida ac in nisl. Pellentesque sit amet
nunc magna. Aliquam mollis vulputate scelerisque. Aenean et scelerisque
tellus. Donec et metus ante. Mauris dapibus lectus eget leo pretium, sit
amet pharetra urna blandit. Suspendisse scelerisque ante mi, quis
imperdiet quam suscipit quis. Mauris felis augue, rutrum a dignissim
non, semper eu turpis. Donec tempus neque non dignissim dignissim. Duis
semper ante risus. Vivamus id consectetur lectus. In enim odio, iaculis
et libero ac, aliquam dapibus mauris. Cras eu velit ornare, tincidunt
dolor quis, pellentesque nulla.

\section{Bibliography}\label{bibliography}

~

\hypertarget{refs}{}
\hypertarget{ref-gammaux5fdesignux5f1995}{}
Gamma, E., 1995. Design patterns: Elements of reusable object-oriented
software, Addison-wesley professional computing series; addison-wesley
professional computing series. Addison-Wesley, Reading, Mass.

%==============================================================================
%==End of content==============================================================
%==============================================================================

%--Acknowledgements-------------------------------------------------------------

\subsection{Acknowledgements}

This work has been partially supported through funding from the National
Science Foundation (\#IIA-1301346)

%--End of Acknowledgements------------------------------------------------------


%--References------------------------------------------------------------------



%--End of references-----------------------------------------------------------

\end{multicols}

%==============================================================================
\end{frame}
\end{document}
